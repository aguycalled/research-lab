\documentclass{mrl}

\title{Sets of spent notes}
\authors{Sarang Noether\footnote{\texttt{sarang.noether@protonmail.com}}}
\affiliations{Monero Research Lab}
\date{\today}

\type{TECHNICAL NOTE}
\ident{MRL-XXXX}

\newtheorem{definition}{Definition}
\newtheorem{example}{Example}

\begin{document}

\begin{abstract}
This technical note formalizes the concept of spent notes using basic set theory. The definition captures a variety of earlier work on identifying such notes. We quantify the effects of this analysis on the Monero blockchain and give a brief overview of possible mitigations.
\end{abstract}

\section{Introduction}
Transactions in Monero generate \textit{notes} (also called \textit{outputs} in literature) destined for a set of recipients by consuming one or more existing notes under the sender's control. For each spent note in the transaction, the sender chooses a collection of arbitrary notes from the blockchain into a \textit{ring}. The transaction includes a proof that for each ring, any of the ring's notes is equiprobable as the spent note. A \textit{tag} (also called a \textit{key image} in literature) is included to ensure that no spent note has been spent in any previous transaction.

It is important for sender anonymity that no note in a ring is otherwise known to have been spent from external information. If a note is known to be spent, an observer can reduce the effective size of the ring as an anonymity set. If this process continues with enough notes, the true spent note may be identified. We stress that Monero notes cannot be linked to the wallet address of the sender, providing an additional layer of protection.

At Monero's launch, senders could choose any ring size, including a ring containing only a single note; this note is obviously the true spend. With this information, it is possible to deduce other spent notes in small rings. Later protocol upgrades added consensus-enforced minimum ring sizes that have increased over time. These increases, as well as a transition to notes with confidential amounts, have all but eliminated the effects of these early trivial rings. However, it is possible to generate more complex sets of rings that together reveal spent notes, even though it may not be possible to identify which transaction spent such a note.

In this technical note, we formally define the idea of a spent note using basic set theory. We show that this definition captures several known methods for spent note identification. Using a tool available to all Monero users, we quantify the occurrence of many spent notes on the Monero blockchain, showing that modern transactions are essentially unaffected by them.

\section{Definition}
Let $\mathcal{N}$ be the (finite) set of all notes on a blockchain. We define a \textit{ring} as a subset of $\mathcal{N}$. A ring containing exactly $n$ elements is an $n$\textit{-ring}. We often use lowercase letters as generic notes.

\begin{definition}
Let $\{R_i\}_{i=1}^n$ be a set of rings. We say each $R_i$ is \textit{spent} if $$\left| \bigcup_{i=1}^n R_i \right| = n.$$ A note is \textit{spent} if it is an element of a spent ring.
\end{definition}

\begin{example}
Let $R = \{a\}$ be a 1-ring. Then the note $a$ (and $R$ itself) is spent.
\end{example}

\begin{example}
Let $R = \{a,b\}$ and $S = \{b,c\}$ and $T = \{a,c\}$ be rings. Then each note (and ring) is spent.
\end{example}

\section{Specific cases}
Earlier work has suggested several classes of spent notes. We review some of them briefly and show how they fit into our definition.

\subsection{Chain reaction}
The so-called \textit{chain reaction} method uses trivial rings to iteratively identify spent notes. This method first marks all 1-rings as spent, and removes the corresponding notes from all other rings. It repeats this process until no 1-rings remain. The initial presentation of this method was identified in the context of an active attack, where the adversary spends many outputs in 1-rings in an attempt to identify honest users' spent notes.

At the end of the iteration process, each spent ring contributes a single unique spent note that was the last such identified note in the ring. This means the collection of all such spent rings matches our definition.

\subsection{Ring repetition}
The so-called \textit{ring repetition} method uses multiple appearances of the same ring to identify spent notes. This method simply identifies a collection of $n$ separate $n$-rings containing the same notes, where we can conclude that the ring is spent. This analysis was initially presented as semi-cooperative attack, where an adversary generates ring repetitions of controlled outputs to signal to other adversarial users that the ring is spent.

This method trivially matches our definition.

\subsection{Subset analysis}
The standard Monero toolset includes an optional blackball tool that scans the blockchain and flags certain classes of spent outputs. In addition to the chain reaction and ring repetition methods, the tool can also perform a \textit{subset analysis}. In this method, the tool iterates over each ring. For each of the $2^n-1$ (nonempty) subsets of an $n$-ring $R$, it counts the number of occurrences of the subset as a standalone ring elsewhere. If the sum of all such counts for subsets of $R$ is exactly $n$, it flags $R$ as spent.

This method trivially matches our definition.

\subsection{Other analysis}
Absent other information, our definition completely captures all spent notes. However, other sources exist in practice that may be used to flag notes as spent, either by attackers or by users who wish to avoid selecting such notes in new rings.

\begin{itemize}
\item \textbf{Chain forks}: In the event of a chain fork, a user may choose to spend the same note on multiple forks. The construction of tags means that each spend of the same note will yield the same tag. An observer who sees distinct rings on multiple forks with the same tag can conclude that the spent note must appear in the intersection of all such rings, which statistically is likely to reveal the spent note. Observe that this analysis is beyond the scope of our definition.
\item \textbf{Note age distribution}: A variety of heuristics exist that may give an adversary a statistical advantage in guessing the spent note in a ring. For example, spend analysis on transparent blockchains suggests that recently-generated notes are more likely to be spent than older notes. We note that in practice, selection of non-spent ring elements according to a distribution matching expected spend patterns easily mitigates the effectiveness of this particular heuristic. There exist other heuristics that we do not consider here. Such heuristics do not inherently provide proof that a given note is spent, and are beyond the scope of our definition. 
\end{itemize}

\section{Mitigations}
It is possible in theory for each user to scan her copy of the blockchain, identify all spent notes using whatever information sources are available, and ensure that she does not choose spent notes as ring members in future transactions. However, a complete set-theoretic characterization using our definition is impractical. Even the use of an integrated blackball tool that performs only a partial analysis may take several hours for a recent snapshot of the Monero blockchain, and would need to be regularly updated for maximal privacy.

Fortunately, the risk to users of spent output identification is negligible. Early chain reaction effects among small rings dissipated quickly early in the Monero blockchain's history. As mandatory minimum ring sizes have increased, the likelihood of an accidental ring union producing a set of spent notes is vanishingly small. While an attacker could generate collections of rings maliciously designed to produce spent notes, a non-cooperating attacker would need to perform intensive computations to detect them; further, the generating attacker can always identify her own controlled outputs regardless of their association with other rings, making such an attack unlikely to be of additional value since it costs the attacker fees.

The number of spent notes produced from a chain fork depends highly on the number of existing notes spent on multiple chains, and requires a large fraction of the existing network to participate. Further, modern selection algorithms for ring members strongly favor newer notes, meaning the effects of a fork dissipate quickly over time. In practice, the combination of these effects renders them impractical.

To quantify these effects, we analyzed the Monero blockchain in October 2018 using the integrated blackball tool. This tool examined several classes of spent notes:
\begin{itemize}
\item notes included in 1-rings
\item notes in repeated rings (discussed above)
\item notes identified by subset analysis (discussed above)
\item notes identified by chain reaction analysis (discussed above)
\end{itemize}
We further classify these notes by whether they use confidential amounts. Modern transactions choose only decoys that use confidential amounts. Table \ref{table:spent} shows the results of this analysis.

\begin{table}[ht]
\begin{center}
\begin{tabular}{rrr}
& Legacy notes & Confidential notes \\
\hline
1-ring & 12147067 & 0 \\
Repeated & 40 & 5 \\
Subset & 5916927 & 0 \\
Chain reaction & 749688 & 0 \\
\hline
Total spent notes & 18813722 & 5 \\
Total notes on chain & 21850122 & 7445622 \\
\end{tabular}
\caption{Spent note analysis for Monero blockchain as of October 2018 using integrated blackball tool}
\label{table:spent}
\end{center}
\end{table}

While the analysis shows that 86\% of all non-confidential notes are identified as spent, 0\% of confidential notes are. Since modern transactions only use the latter type as decoys, the effects of spent note analysis on anonymity is completely negligible.

\section{Conclusion}
We have presented a simple set-theoretic definition that completely characterizes spent notes on the Monero blockchain given only set information about the ring elements themselves. This definition captures and generalizes other analysis presented elsewhere. While this definition does not address external information from sources like forked chains or temporal analysis, it offers insight into the selection of notes for optimal spend anonymity. While a complete analysis of all spent notes on the Monero blockchain is computationally infeasible, we quantified several known classes of spent outputs and determined that modern transactions are unaffected by them.

\end{document}
